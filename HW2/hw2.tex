\documentclass{article}

\begin{document}

\section*{Question 4}
\subsection*{Prove, mathematically, that both roots are negative.}

% Enter your answer to Question 5 below this line
See the equations in Question 5 and let $a=1$ and $c=1$, then we can see:\\
$x_1 = ( -b + sqrt(b*b - 4) ) / 2$\\
$x_2 = ( -b - sqrt(b*b - 4) ) / 2$\\
when $b$ in $(1e1, 1e2... 1e15)$, and $b>=10$\\
so we can get:\\
$b*b>(b*b -4)$\\
$-b + sqrt(b*b - 4)<0$ and $-b - sqrt(b*b - 4)<0$\\
Thus, $x_1$, $x_2$ are both negative.

\section*{Question 5}
\subsection*{What is $x_1 x_2$ in terms of $a, b, c$?}

% Enter your answer to Question 5 below this line
$x_1 = ( -b + sqrt(b*b - 4*a*c) ) / (2*a)$\\
$x_2 = ( -b - sqrt(b*b - 4*a*c) ) / (2*a)$

\section*{Question 6}
\subsection*{Look at \texttt{roots.txt}.  What do you notice about one of the
             roots?}

% Enter your answer to Question 6 below this line
1. The precision of  of the $x_1$ are  smaller than 17.\\
2. The $x_1$ becoms 0.000000 after $b>=9$, but accroding to Qestion 4, the value of it should be negative.

\section*{Question 7}
\subsection*{Look at the formula for the quadratic equation for the
             solution $x_1$. For fixed $a$ and $c$, how do the magnitudes of
             terms in the numerator compare as b gets large?}

% Enter your answer to Question 7 below this line
1. The precision of $x_1$ gets smaller as $b$ gets larger.\\
2. Mathematically, as $b$ gets larger, $x_1$ becomes bigger and $x_2$ becomes smaller. In the case of $a=1 c=1$, $x_1$ should become bigger but not bigger than 0.

\section*{Question 8}
\subsection*{Given your analysis in 8, discuss what you think is happening in
             the finite precision calculation of $x_1$?}

% Enter your answer to Question 8 below this line
1. Since $x_1$ is the double type in the c file, so it's width is 64 bits and it only contains 17 precision digits. When the number of decimal exceeds 17 decimal digits, machine memory overflows. During the calculating process, $b-sqrt(b^2-4ac)$ is a very small number after $b>=1e9$ and the percision is reduced dramatically. Thus, we can't get the specific the decimal digits even before 16.\\ 
2. Also after the percision exceeding 17 digits, machine can not operate it. After $b>1e8$, $x_1$ becomes zero.


\end{document}
